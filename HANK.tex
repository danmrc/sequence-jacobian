\documentclass[11pt,a4paper]{article}
\usepackage[top=30mm,bottom=30mm,left=25mm,right=25mm]{geometry}
\usepackage{newcent} 							% New Century Schoolbook font
\usepackage[english]{babel}  		   		   	% language and its typo rules
\usepackage[utf8]{inputenc}		 	   		   	% input accented characters
\usepackage{xcolor}               	   		   	% change colors
\usepackage{mathtools,amssymb}					% for math characters, also loads amsmath
\usepackage{enumitem}							% for enumerate
\usepackage{comment}

\definecolor{myb}{rgb}{0,0.4470,0.7410}        	% custom blue color
\definecolor{peach}{rgb}{0.85,0.325,0.0980}    	% custom orange color


\begin{document}
\title{Consumption Tax in HANK}
\author{Yvan Becard}
%\date{}
\maketitle



\section{One-Asset HANK as in Auclert et al. (2021)}
\label{hank}
In this model, $\tau_t$ is a \textit{tax}, proportional to households' labor productivity $e_{it}$: $\tau_t\bar{\tau}(e_{it})$.

\paragraph{Households}
Continuum of households $i$. Utility function
\begin{equation*}
E_0\sum_{t=0}^{\infty}\beta^t\left\{\frac{c_{it}^{1-\sigma}}{1-\sigma}-\varphi\frac{n_{it}^{1+\nu}}{1+\nu}\right\}.
\end{equation*}
Budget constraint and borrowing constraint
\begin{align*}
&c_{it}+b_{it}=w_te_{it}n_{it}+(1+r_t)b_{it-1}-\tau_t\bar{\tau}(e_{it})+d_t\bar{d}(e_{it}),\\
&b_{it}\geq\underline{b}.
\end{align*}
Lagrangian
\begin{equation*}
\mathcal{L}_i=E_0\sum_{t=0}^{\infty}\beta^t\left\lbrace \frac{c_{it}^{1-\sigma}}{1-\sigma}-\varphi\frac{n_{it}^{1+\nu}}{1+\nu}+\lambda_{it}\left[w_te_{it}n_{it}+(1+r_t)b_{it-1}-\tau_t\bar{\tau}(e_{it})+d_t\bar{d}(e_{it})-c_{it}-b_{it}\right]\right\rbrace.
\end{equation*}
Optimal labor supply and consumption-saving decisions for unconstrained households
\begin{align*}
%w_t&=\frac{\varphi n_{it}^{\nu}}{e_{it}c_{it}^{-\sigma}}.\\
n_{it}&=\left(\frac{w_te_{it}}{\varphi c_{it}^{\sigma}}\right)^{\frac{1}{\nu}}.\\
c_{it}^{-\sigma}&=\beta E_t(1+r_{t+1})c_{it+1}^{-\sigma}.
\end{align*}
When the borrowing constraint binds, $b_{it}=\underline{b}$, the budget constraint rewrites as
\begin{equation*}
c_{it}+\underline{b}=w_te_{it}n_{it}+(1+r_t)b_{it-1}-\tau_t\bar{\tau}(e_{it})+d_t\bar{d}(e_{it}).
\end{equation*}
Plug in the labor supply condition
\begin{equation*}
c_{it}+\underline{b}=w_te_{it}\left(\frac{w_te_{it}}{\varphi c_{it}^{\sigma}}\right)^{\frac{1}{\nu}}+(1+r_t)b_{it-1}-\tau_t\bar{\tau}(e_{it})+d_t\bar{d}(e_{it}).
\end{equation*}
%Lagrangian
%\begin{equation*}
%\mathcal{L}_i=E_0\sum_{t=0}^{\infty}\beta^t\left\lbrace \frac{c_{it}^{1-\sigma}}{1-\sigma}-\varphi\frac{n_{it}^{1+\nu}}{1+\nu}-\lambda_{it}\left[c_{it}+\underline{b}_{}-(1+r_t)b_{it-1}-w_te_{it}n_{it}+\tau_t\bar{\tau}(e_{it})-d_t\bar{d}(e_{it})\right]\right\rbrace.
%\end{equation*}


\paragraph{Firms}
Continuum of intermediate firms $j$. Production function
\begin{equation*}
y_{jt}=Z_tn_{jt}.
\end{equation*}
Cost minimization $-w_tn_{jt}+mc_{jt}[Z_tn_{jt}-y_{jt}]$ yields labor demand
\begin{equation*}
mc_{jt}=mc_t=\frac{w_t}{Z_t}.
\end{equation*}
Quadratic adjustment cost
\begin{equation*}
\psi(p_{jt},p_{jt-1})=\frac{\mu}{\mu-1}\frac{1}{2\kappa}\left[\log\left(\frac{p_{jt}}{p_{jt-1}}\right)\right]^2Y_t.
\end{equation*}
Profit
\begin{equation*}
d_{jt}=\frac{p_{jt}}{P_t}y_{jt}-w_tn_{jt}-\frac{\mu}{\mu-1}\frac{1}{2\kappa}\left[\log\left(\frac{p_{jt}}{p_{jt-1}}\right)\right]^2Y_t.
\end{equation*}
Profit maximization, where $M_{t}\equiv\beta^t c_{t}^{-\sigma}$
\begin{equation*}
\begin{gathered}
\max_{p_{jt}}E_0\sum_{t=0}^{\infty}M_{t}\left[\frac{p_{jt}}{P_t}y_{jt}-w_tn_{jt}-\psi(p_{jt},p_{jt-1})\right]\\
\text{subject to}\quad y_{jt}=Z_tn_{jt},\,\, y_{jt}=\left(\frac{p_{jt}}{P_t}\right)^{-\frac{\mu}{\mu-1}}Y_t,\,\,\text{and } \psi(p_{jt},p_{jt-1})=\frac{\mu}{\mu-1}\frac{1}{2\kappa}\left[\log\left(\frac{p_{jt}}{p_{jt-1}}\right)\right]^2Y_t.
\end{gathered}
\end{equation*}
Plug the constraints into the objective
\begin{equation*}
\max_{p_{jt}}E_0\sum_{t=0}^{\infty}M_{t}\left[\left(\frac{p_{jt}}{P_t}\right)^{-\frac{1}{\mu-1}}Y_t-w_t\left(\frac{p_{jt}}{P_t}\right)^{-\frac{\mu}{\mu-1}}\frac{Y_t}{Z_t}-\frac{\mu}{\mu-1}\frac{1}{2\kappa}\left[\log(p_{jt})-\log(p_{jt-1})\right]^2Y_t\right].
\end{equation*}
Phillips curve (ie FOC with respect to $p_{jt}$, after imposing symmetric equilibrium $p_{jt}=P_t$, defining inflation $1+\pi_t\equiv P_{t}/P_{t-1}$, and aggregate discount factor $M_{t+1}/M_t=(1+r_{t+1})^{-1}$)
\begin{equation*}
\log(1+\pi_t)=\kappa\left(\frac{w_t}{Z_t}-\frac{1}{\mu}\right)+E_t\frac{1}{1+r_{t+1}}\frac{Y_{t+1}}{Y_t}\log(1+\pi_{t+1}).
\end{equation*}
Aggregate dividends
\begin{equation*}
d_t=Y_t-w_tN_t-\psi_t.
\end{equation*}


\paragraph{Government}
Government budget constraint: the tax finances interest payment on bonds
\begin{equation*}
\tau_t=r_tB.
\end{equation*}
Monetary policy
\begin{equation*}
i_t=r_t^*+\phi_{\pi}\pi_t.
\end{equation*}
Fisher equation
\begin{equation*}
1+r_t=\frac{1+i_{t-1}}{1+\pi_t}.
\end{equation*}


\paragraph{Market clearing} Aggregate production function and resource constraint
\begin{equation*}
Y_t=Z_tN_t;\quad Y_t=C_t+\psi_t\quad\text{where }C_t=\int_i c_{it}di.
\end{equation*}
Clearing in the bond and labor markets
\begin{equation*}
B_t=\int_ib_{it}di;\quad N_t=\int_ie_{it}n_{it}di.
\end{equation*}




\section{One-Asset HANK with Exogenous Uniform Cash Transfers}
\label{hanktransfer}
Same model as in Section \ref{hank}, except that $\tau_t$ is no longer a proportional \textit{tax} but instead becomes an exogenous, lump-sum \textit{transfer}. Government debt $B$ becomes a time-varying endogenous variable, $B_t$.


\paragraph{Households}
Same utility function, borrowing constraint, and FOCs as before. Only transfers in the budget constraint change
\begin{align*}
c_{it}+b_{it}=w_te_{it}n_{it}+(1+r_t)b_{it-1}+\tau_t+d_t\bar{d}(e_{it}).
\end{align*}


\paragraph{Firms} Same as before

\paragraph{Government} Government budget constraint: $\tau_t$ is now a transfer
\begin{equation*}
\tau_t+(1+r_t)B_{t-1}=B_t.
\end{equation*}
In steady state, this implies $\tau=-rB$.
%Same monetary policy and Fisher equation
%\begin{equation*}
%i_t=r_t^*+\phi\pi_t;\quad 1+r_t=\frac{1+i_{t-1}}{1+\pi_t}.
%\end{equation*}

\paragraph{Market clearing} Same as before




\section{One-Asset HANK with Transfers and Taxes}
\label{hanktax}
Same model as in Section \ref{hanktransfer}, but now we add exogenous consumption and labor taxes.


\paragraph{Households}
Same utility function as before. Budget and borrowing constraints
\begin{align*}
&(1+\tau_{ct})c_{it}+b_{it}=(1-\tau_{n})w_te_{it}n_{it}+(1+r_t)b_{it-1}+\tau_t+d_t\bar{d}(e_{it}),\\
&b_{it}\geq\underline{b}.
\end{align*}
%Lagrangian
%\begin{dmath*}
%\mathcal{L}_i=E_0\sum_{t=0}^{\infty}\beta^t\left\lbrace \frac{c_{it}^{1-\sigma}}{1-\sigma}-\varphi\frac{n_{it}^{1+\nu}}{1+\nu}+\lambda_{it}\left[(1-\tau_n)w_te_{it}n_{it}+(1+r_t)b_{it-1}-\tau_t\bar{\tau}(e_{it})+d_t\bar{d}(e_{it})-(1+\tau_{ct})c_{it}-b_{it}\right]\right\rbrace.
%\end{dmath*}
Optimal labor supply and consumption-saving decisions
\begin{align*}
n_{it}&=\left[\frac{(1-\tau_n)w_te_{it}}{(1+\tau_{ct})\varphi c_{it}^{\sigma}}\right]^{\frac{1}{\nu}}.\\
c_{it}^{-\sigma}&=\beta E_t\frac{1+\tau_{ct}}{1+\tau_{ct+1}}(1+r_{t+1})c_{it+1}^{-\sigma}.
\end{align*}
Budget constraint when borrowing constraint binds
\begin{equation*}
(1+\tau_{ct})c_{it}+\underline{b}=(1-\tau_n)w_te_{it}n_{it}+(1+r_t)b_{it-1}+\tau_t+d_t\bar{d}(e_{it}).
\end{equation*}
Plug in the labor supply condition
\begin{equation*}
(1+\tau_{ct})c_{it}+\underline{b}=(1-\tau_n)w_te_{it}\left[\frac{(1-\tau_n)w_te_{it}}{(1+\tau_{ct})\varphi c_{it}^{\sigma}}\right]^{\frac{1}{\nu}}+(1+r_t)b_{it-1}+\tau_t+d_t\bar{d}(e_{it}).
\end{equation*}



\paragraph{Firms} Same as before


\paragraph{Government}
Government budget constraint
\begin{equation*}
\tau_t+(1+r_t)B_{t-1}=\tau_{ct}C_t+\tau_{n}W_tN_t+B_t.
\end{equation*}
In steady state, this implies $\tau=-rB+\tau_cC+\tau_nWN$.
%Same monetary policy and Fisher equation
%\begin{equation*}
%i_t=r_t^*+\phi\pi_t;\quad 1+r_t=\frac{1+i_{t-1}}{1+\pi_t}.
%\end{equation*}


\paragraph{Market clearing} Same as before




\section{One-Asset HANK with Sticky Wage}
\label{hanksticky}
Same model as in section \ref{hanktransfer}, except that the labor supply equation is replaced by a wage Phillips curve.


\paragraph{Unions}
Continuum of unions $k$. Every household supplies every labor types, so each union represents all households. Union $k$ chooses a wage $W_{kt}$ to maximize
\begin{equation*}
\begin{gathered}
E_0\sum_{t=0}^{\infty}\beta^t\left\{\int_i\left[\frac{c_{it}^{1-\sigma}}{1-\sigma}-\varphi\frac{n_{it}^{1+\nu}}{1+\nu}\right]di-\psi_w(W_{kt},W_{kt-1})\right\}\\
\text{subject to}\quad N_{kt}=\left(\frac{W_{kt}}{W_t}\right)^{-\frac{\mu_w}{\mu_w-1}}N_t\quad\text{and}\quad\psi_w(W_{kt},W_{kt-1})=\frac{\mu_w}{\mu_w-1}\frac{1}{2\kappa_w}\left[\log\left(\frac{W_{kt}}{W_{kt-1}}\right)\right]^2.
\end{gathered}
\end{equation*}
Household $i$'s total real earnings are
\begin{align*}
z_{it}&=(1-\tau_n)\frac{W_{t}}{P_t}e_{it}n_{it}\lambda\\
&=(1-\tau_n)\left(\frac{1}{P_t}\int_0^1W_{k,t}e_{it}n_{ikt}dk\right)\\
&=(1-\tau_n)\left[\frac{1}{P_t}\int_0^1W_{k,t}e_{it}\left(\frac{W_{kt}}{W_t}\right)^{-\frac{\mu_w}{\mu_w-1}}N_tdk\right].
\end{align*}
The envelope theorem implies that we can evaluate indirect utility as if all income from the union wage change is consumed. That means $\frac{\partial c_{it}}{\partial W_{kt}}=\frac{\partial z_{it}}{\partial W_{kt}}$, where
\begin{align*}
\frac{\partial z_{it}}{\partial W_{kt}}&=(1-\tau_n)\frac{e_{it}}{P_t}\left(1-\frac{\mu_w}{\mu_w-1}\right)N_{kt}.
\end{align*}
Household $i$'s total hours worked are
\begin{equation*}
n_{it}=\int_0^1\left(\frac{W_{kt}}{W_t}\right)^{-\frac{\mu_w}{\mu_w-1}}N_tdk.
\end{equation*}
Hours fall when $W_{kt}$ increases 
\begin{equation*}
\frac{\partial n_{it}}{\partial W_{kt}}=-\frac{\mu_w}{\mu_w-1}\frac{N_{kt}}{W_{kt}}.
\end{equation*}
Combining everything, the first-order condition of the union with respect to $W_{kt}$ is
\begin{equation*}
\begin{split}
0&=\int_i N_{kt}\left\{(1-\tau_n)\frac{e_{it}}{P_t}\left(1-\frac{\mu_w}{\mu_w-1}\right)c_{it}^{-\sigma}+\varphi\frac{\mu_w}{\mu_w-1}\frac{1}{W_{kt}}n_{it}^{\nu}\right\}di\\
&-\frac{\mu_w}{\mu_w-1}\frac{1}{\kappa_w}\log\left(\frac{W_{kt}}{W_{kt-1}}\right)\frac{1}{W_{kt}}+\beta\frac{\mu_w}{\mu_w-1}\frac{1}{\kappa_w}E_t\log\left(\frac{W_{kt+1}}{W_{kt}}\right)\frac{1}{W_{kt}}.
\end{split}
\end{equation*}
In a symmetric equilibrium, all unions set the same wage, so $W_{kt}=W_t$ and $n_{it}=N_{kt}=N_t$. Define wage inflation $1+\pi_t^w=\frac{w_t}{w_{t-1}}$ and obtain the aggregate wage Phillips curve
\begin{equation*}
\log(1+\pi_t^w)=\kappa_w\left(\varphi N_{t}^{1+\nu}-\frac{(1-\tau_n)w_tN_t}{\mu_w}\int_i e_{it}c_{it}^{-\sigma}di\right)+\beta E_t\log(1+\pi_{t+1}^w).
\end{equation*}
In steady state, the wage Phillips curve gives us, solving for $\varphi$
\begin{equation*}
\varphi=\frac{(1-\tau_n)wN^{-\nu}}{\mu_w}\int_{i}e_{i}c_{i}^{-\sigma}di.
\end{equation*}



\section{One-Asset HANK with Sticky Wage and Consumption Tax TO CHECK}
Same model as in section \ref{hanktax}, except that the labor supply equation is replaced by a wage Phillips curve in which we have consumption taxes.


\paragraph{Unions}
Continuum of unions $k$. Every household supplies every labor types, so each union represents all households. Union $k$ chooses a wage $W_{kt}$ to maximize
\begin{equation*}
\begin{gathered}
E_0\sum_{t=0}^{\infty}\beta^t\left\{\int_i\left[\frac{c_{it}^{1-\sigma}}{1-\sigma}-\varphi\frac{n_{it}^{1+\nu}}{1+\nu}\right]di-\psi_w(W_{kt},W_{kt-1})\right\}\\
\text{subject to}\quad N_{kt}=\left(\frac{W_{kt}}{W_t}\right)^{-\frac{\mu_w}{\mu_w-1}}N_t\quad\text{and}\quad\psi_w(W_{kt},W_{kt-1})=\frac{\mu_w}{\mu_w-1}\frac{1}{2\kappa_w}\left[\log\left(\frac{W_{kt}}{W_{kt-1}}\right)\right]^2.
\end{gathered}
\end{equation*}
Household $i$'s total real earnings are
\begin{align*}
z_{it}&=\frac{1-\tau_n}{1+\tau_{ct}}\frac{W_{t}}{P_t}e_{it}n_{it}\lambda\\
&=\frac{1-\tau_n}{1+\tau_{ct}}\left(\frac{1}{P_t}\int_0^1W_{k,t}e_{it}n_{ikt}dk\right)\\
&=\frac{1-\tau_n}{1+\tau_{ct}}\left[\frac{1}{P_t}\int_0^1W_{k,t}e_{it}\left(\frac{W_{kt}}{W_t}\right)^{-\frac{\mu_w}{\mu_w-1}}N_tdk\right].
\end{align*}
The envelope theorem implies that we can evaluate indirect utility as if all income from the union wage change is consumed. That means $\frac{\partial c_{it}}{\partial W_{kt}}=\frac{\partial z_{it}}{\partial W_{kt}}$, where
\begin{align*}
\frac{\partial z_{it}}{\partial W_{kt}}&=\frac{1-\tau_n}{1+\tau_{ct}}\frac{e_{it}}{P_t}\left(1-\frac{\mu_w}{\mu_w-1}\right)N_{kt}.
\end{align*}
Household $i$'s total hours worked are
\begin{equation*}
n_{it}=\int_0^1\left(\frac{W_{kt}}{W_t}\right)^{-\frac{\mu_w}{\mu_w-1}}N_tdk.
\end{equation*}
Hours fall when $W_{kt}$ increases 
\begin{equation*}
\frac{\partial n_{it}}{\partial W_{kt}}=-\frac{\mu_w}{\mu_w-1}\frac{N_{kt}}{W_{kt}}.
\end{equation*}
Combining everything, the first-order condition of the union with respect to $W_{kt}$ is
\begin{equation*}
\begin{split}
0&=\int_i N_{kt}\left\{\frac{1-\tau_n}{1+\tau_{ct}}\frac{e_{it}}{P_t}\left(1-\frac{\mu_w}{\mu_w-1}\right)c_{it}^{-\sigma}+\varphi\frac{\mu_w}{\mu_w-1}\frac{1}{W_{kt}}n_{it}^{\nu}\right\}di\\
&-\frac{\mu_w}{\mu_w-1}\frac{1}{\kappa_w}\log\left(\frac{W_{kt}}{W_{kt-1}}\right)\frac{1}{W_{kt}}+\beta\frac{\mu_w}{\mu_w-1}\frac{1}{\kappa_w}E_t\log\left(\frac{W_{kt+1}}{W_{kt}}\right)\frac{1}{W_{kt}}.
\end{split}
\end{equation*}
In a symmetric equilibrium, all unions set the same wage, so $W_{kt}=W_t$ and $n_{it}=N_{kt}=N_t$. Define wage inflation $1+\pi_t^w=\frac{w_t}{w_{t-1}}$ and obtain the aggregate wage Phillips curve
\begin{equation*}
\log(1+\pi_t^w)=\kappa_w\left(\varphi N_{t}^{1+\nu}-\frac{(1-\tau_n)w_tN_t}{(1+\tau_{ct})\mu_w}\int_i e_{it}c_{it}^{-\sigma}di\right)+\beta E_t\log(1+\pi_{t+1}^w).
\end{equation*}
In steady state, the wage Phillips curve gives us, solving for $\varphi$
\begin{equation*}
\varphi=\frac{(1-\tau_n)wN^{-\nu}}{(1+\tau_c)\mu_w}\int_{i}e_{i}c_{i}^{-\sigma}di.
\end{equation*}




\end{document}
